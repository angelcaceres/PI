\documentclass[12pt]{article}
\usepackage[utf8]{inputenc}
\usepackage{amsfonts}
\usepackage[spanish]{babel}
\usepackage{pgfplots}
\pgfplotsset{compat=1.15}
\title{Proyecto inicial para la materia de Física Clásica}
\date{2019\\ Abril}
\author{Ángel Cáceres Licona, UAM Cuajimalpa}
\begin{document}
\maketitle

\section*{Cantidades cinemáticas}

La mecánica tiene como objetivo la descripción del movimiento y sus causas. Describimos el movimiento de una partícula a travez de un vector en $\mathbb{R}^3$, 

\begin{center}
    \begin{tikzpicture}
        \begin{axis}[view={60}{30},
                     axis lines=center,axis on top,
                     xlabel=$x$,ylabel=$y$,zlabel=$z$,
                     no marks,axis equal,
                     xtick={0},ytick={0},ztick={0},
                     xmin=-1,xmax=4,ymin=-1,ymax=4,zmin=-1,zmax=4]
          \addplot3+[no markers,samples=51, samples y=0,domain=-pi:pi/2,variable=\t]
                                           ({exp(\t)*cos(\t r)},{exp(\t)*sin(\t r)},{exp(\t)});
       \end{axis}
     \end{tikzpicture}
        
\end{center}
con $r \in \mathbb{R}^3$.

La trayectoria es una curva paramétrica. En este caso con parámetro $t$.
\begin{eqnarray}
    \vec{r}(t) = x^1(t)\hat {e_1}+ x^2(t) \hat{e_2} +x^3(t)\hat{e_3},\\
    \vec{r}(t) = \sum_{i=1}^{3}x^i(t)\hat{e_i}
\end{eqnarray}

\section*{Suma de Einstein}
Existe una notación más compacta llamada convención de Einstein.
\begin{eqnarray}
    x^i(t)\hat {e_i},
\end{eqnarray}
con $i=1,2,3$.

\subsection{Nota vectorial}
Sea $\vec{A} \in \mathbb{R^3}; \vec{A}=A_1^i\hat{e}_i$ donde $\hat{e}_i$ es la base cartesiana. Las componentes seobtienen con el producto punto:
\begin{eqnarray}
    A^j=\vec{A}\cdot\hat{e_j},\\
    A^j=(A^j\hat{e}_i)\cdot\hat{e}_j,\label{sumaE}
\end{eqnarray}
donde, en (\ref{sumaE}), lo que está entre paréntesis es la suma de Einstein.
\begin{eqnarray}
    A^j=A^i\hat{e}_i\cdot\hat{e}_j,
\end{eqnarray}

\section*{Sistemas de coordenadas}

\section*{Cambios de coordenadas}

\section*{Curvas paramétricas}

\section*{Superficies paramétricas}

\section*{Relaciones entre cantidades cinemáticas y geométricas}

\subsection*{Mood}


\end{document}
